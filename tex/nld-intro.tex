Nonlinear Dynamic Methods are based on the results of many dynamic analyses, which relate the seismic response of a structure, represented by an Engineering Demand Parameter (EDP), like maximum top displacement, maximum inter-storey drift ratio, maximum top drift etc., to the Intensity Measure Level (IML) of the input accelerograms. 
Many methods exists in literature to perform a series of dynamic analysis and to post-process the results in order to derive fragility curves. Some of them treat a single building to estimate directly the median seismic intensity value corresponding to the attainment of different damage state threshold (limit state), and the corresponding dispersion (Vamvatsikos and Cornell, 2002, Ellingwood and Kinali, 2009). Others treat a class of buildings, and lead to the evaluation of the probabilities of different damage states for a series of IMLs and thus to the set up of a damage probability matrix (Singhal and Kiremidjian, 1996, Silva et al., 2013).

The last approach have been implemented in the DPM-based procedure, explained in section \ref{sec:DPM}, from the point of view of the necessary scientific background behind and their step-by-step implementation in the python script.