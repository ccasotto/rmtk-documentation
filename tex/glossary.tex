% OpenQuake Book Glossary 
% To cite a glossary element in a document:
%	\gls{seismicsourcedata}
%	\Gls{seismicsourcedata} - First initial is uppercase
%	\GLS{seismicsourcedata} - All initials are uppercase
%	\glspl{seismicsourcedata} - Plural
% To process the glossary:
% 	makeglossaries oqb

%
% ------- A
%
% ------- B
\newglossaryentry{branch}{
	name = branch,
	plural= branches,
	description={
	The simplest element in a logic tree; it belongs to a 
	\gls{branchset} where it represents one possible option among a finite 
	number of alternatives. A branch is associated with a weight 
	value \citep{scherbaum2011} if the \gls{branchset} represents the 
	epistemic uncertainty on a parameter or a model when the \gls{branchset} 
	is used to specify alternative models (e.g. district \glspl{acr:mfd})
	}
}
%
% ------- C
\newacronym{cpsha}{cPSHA}{Classical PSHA}
\newglossaryentry{configurationfile}{
	name =  configuration file,
	description = {
	Usually the file containing the information necessary to run a calculation
	in OpenQuake
	}
}
%
% ------- D
%
% ------- E
\newacronym{acr:erf}{ERF}{Earthquake\- Rup\-ture\- Forecast}
\newacronym{acr:epsha}{ePSHA}{Event-based PSHA}
%
\newglossaryentry{earthquakeruptureforecast}{
	name = earthquake rupture forecast,
	description={
	A list of all possible ruptures generated by all the sources included 
	in a seismic source model. Each element in the list contains: the rupture 
	geometry and the rupture probability of occurrence in a given time span. 
	%
	See also the definition available on the 
	\href{http://www.opensha.org/glossary-earthquakeRuptureForecast}
	{OpenSHA website}}
}
\newglossaryentry{earthquakeruptureforecastcalculator}{
	name = earthquake rupture forecast calculator,
	description={
	Calculator producing a \gls{seismicsourcemodel} from a 
	\gls{seismicsourcelogictree} 
	}
}
%
%
% ------- F
%
% ------- G
\newacronym{acr:gem}{GEM}{Global Earthquake Model}
\newacronym{acr:gmpe}{GMPE}{Ground Motion Prediction Equation}
\newacronym{acr:gsim}{GSIM}{Ground Shaking Intensity Model}
\newacronym{acr:gmm}{GMM}{Ground Motion Model}

\newglossaryentry{groundmotionfield}{
	name = ground-motion field,
	description={An object describing the geographic distribution around 
	a rupture of a ground motion intensity measure}
}
\newglossaryentry{groundmotionmodel}{
	name = ground-motion model,
	description={An object that given a rupture with specific properties
	computes the expected ground motion at the given site. In simplest case 
	a ground motion model corresponds to a \gls{groundmotionpredictioneq}. 
	In case of complex PSHA input models, the produced ground motion models 
	contains a set of \glspl{acr:gmpe}, one for each tectonic region considered.
	}
}
\newglossaryentry{groundmotionpredictioneq}{
	name = ground-motion prediction equation,
	description={
		An equation that - given some fundamental parameters characterizing 
		the source, the propagation path and the site (in the simplest 
		case magnitude, distance and V$_\text{S,30}$) - computes the 
		value $GM$ of a (scalar) ground motion intensity parameter.
	}
}
%
% ------- I 
\newacronym{acr:imt}{IMT}{Intensity Measure Type}
\newglossaryentry{investigationtime}{
	name = investigation time,
	description={The time interval considered to calculate hazard; usually 
	it corresponds to 50 years}
}
%
% ------- L
\newglossaryentry{logictree}{
	name = logic tree,
	description={Data structure used to systematically describe uncertainties
	on parameters and models used in a PSHA study}
}
%
% ------- M
%
% ------- O
\newacronym{acr:oq}{OQ}{OpenQuake}
\newacronym{acr:oqe}{OQ-engine}{OpenQuake-engine}
\newacronym{acr:oqhl}{OQ-hazardlib}{OpenQuake hazard library}
\newacronym{acr:oqrl}{OQ-risklib}{OpenQuake risk library}
%
% ------- N
\newacronym{acr:nrml}{NRML}{Natural hazard Risk Markup Language}
%
% ------- P
\newacronym{acr:pga}{PGA}{Peak Ground Acceleration}
\newacronym{acr:pgv}{PGV}{Peak Ground Velocity}
\newacronym{acr:psha}{PSHA}{Probabilistic Seismic Hazard Analysis}
\newacronym{acr:peer}{PEER}{Pacific Earthquake Engineering Center}
%
\newglossaryentry{psha}{
	name = probabilistic seismic hazard analysis, 
	description={A methodology to compute seismic hazard which takes into 
	account the contributions coming from all the sources of engineering 
    importance for a specified site}	
}
%
% ------- R
%
% ------- S
\newglossaryentry{softwarequalityassurance}{
	name = software quality assurance,
	description={
    Software quality assurance (SQA) consists of a means of monitoring 
    the software engineering processes and methods used to ensure quality.
    The methods by which this is accomplished are many and varied, and 
    may include ensuring conformance to one or more standards, such as 
    ISO 9000 or a model such as CMMI.
    SQA encompasses the entire software development process, which 
    includes processes such as requirements definition, software design, 
    coding, source code control, code reviews, software configuration 
    management, testing, release management, and product integration. 
    SQA is organized into goals, commitments, abilities, activities,
    measurements, and verifications.
	}
}
\newacronym{acr:sa}{S$_a$}{Spectral Acceleration}
%
% ------- T
%
% ------- U
\newacronym{acr:usgs}{USGS}{United States Geological Survey}
%
% ------- V 