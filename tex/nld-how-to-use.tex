To start using the nonlinear dynamic method a command line text editor should be used to enter manually the folder location where the RMTK has been saved. The user should add the path \textit{/RMTK/Vulnerability/NDP}, where the nonlinear dynamic method script is located, as shown in the example below:

\begin{Verbatim}[frame=single, commandchars=\\\{\}, samepage=true]
cd path/to/rmtk/folder/RMTK
\end{Verbatim}

From the text editor iPython browser page can be opened with the following command line:

\begin{Verbatim}[frame=single, commandchars=\\\{\}, samepage=true]
ipython-2.7 notebook --pylab=inline
\end{Verbatim}

Once the iPython page is opened on the browser, the python scripts contained in the RMTK directory will be visible. The file \textit{NDM.ipynb} should be selected to start the calculations.

In the initial section of the script "Define Options" the user needs to set the options and to enter the input corresponding to the defined options in the folder \textit{NDP/input}. In section~\ref{subsec:NDMoptions} the alternatives values that the initial variables can assume and their meaning are described in detail, while the parameters to be inserted in the input files are fully described in section~\ref{subsec:NDMinputs}.