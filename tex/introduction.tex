\section{Introduction}
The Risk Modeller's Toolkit (or RMTK) is a Python 2.7 library of functions written by scientists at the GEM Model Facility, which is intended to provide scientists and engineers with the tools to help create the vulnerability input models that go into the OpenQuake risk engine. The intention of this software is to provide scientists and engineers with the means to apply some of the most commonly used algorithms for preparing vulnerability models using structural analysis data. The current approach consists of the derivation of fragility curves and the combination with a damage-to-loss function for the definition of a discrete vulnerability function. The just released GEM analytical vulnerability guidelines have been integrated in this tool and some of the methodologies indicated have been already implemented in the library. In forthcoming versions will hope to make available more methodologies for the process indicated here, and to integrate new functionalities for i) building structural models of different levels of complexity within the RMTK in combination with a structural analysis software, ii) running dynamic and static analysis within the RMTK in combination with a structural analysis software, iii) deriving vulnerability curves directly applying engineering demand parameters-to-loss functions to structural analysis results.

This section provides a description of the methods currently implemented in the RMTK, and an initial presentation of the input and output files is provided. In the following sections, the contents and structure of these files are discussed in detail.

\subsection{Getting Started and Running the Software}
Install notebook and dependencies. go to folder in the command line (RMTK/Vulnerability/name-of-the-procedure)